\documentclass[a4paper,10pt]{scrartcl}
\usepackage[utf8]{inputenc}
\usepackage[ngerman]{babel}
\usepackage{amsmath}
\usepackage{amsfonts}
\usepackage{amsthm}
\usepackage{geometry}
\usepackage{multicol}
\usepackage{minted}
\geometry{a4paper,left=30mm,right=30mm, top=3cm, bottom=2cm} 

\newcommand{\underfat}[1]{\underline{\textbf{#1}}}
\newcommand{\theuebungszettel}{3}

\parindent0pt

\begin{document}

\begin{center}
  \begin{huge}
    \underfat{Fortgeschrittene funktionale}\\
    \underfat{Programmierung in Haskell}\\
  \end{huge}
\begin{LARGE}
\textbf{Übungszettel \theuebungszettel}
\end{LARGE}
\end{center}
\section*{Aufgabe \theuebungszettel.1:}
In der Vorlesung wurde MaybeT explizit vorgestellt. Erstellen sie einen Monad-Transformer EitherT für Either. Zur Erinnerung: Either ist definiert als:
\begin{minted}{haskell}
data Either a b = Left a
                | Right b
\end{minted}
Erstellen sie hierzu die Instanzen für
\begin{itemize}
 \item Functor
       \begin{minted}{haskell}
  instance Functor f => Functor (EitherT f) where
      fmap :: (a -> b) -> (EitherT f) a -> (EitherT f) b
       \end{minted}
 \item Applicative
       \begin{minted}{haskell}
  instance Applicative f => Applicative (EitherT f) where
      pure  :: a -> f a
      (<*>) :: f (a -> b) -> f a -> f b
       \end{minted}
 \item Monad
       \begin{minted}{haskell}
  instance Monad m => Monad (EitherT m) where
      return :: a -> (EitherT m) a
      (>>=)  :: (EitherT m) a -> (a -> (EitherT m) b) -> (EitherT m) b
       \end{minted}
\end{itemize}
\section*{Aufgabe \theuebungszettel.2:}
In dieser Aufgabe geht es um die Verwendung eines Monad-Stacks. Hierzu schreiben sie ein (sehr!) simples Spiel:\\
Durch drücken von \texttt{u} (up) bzw. \texttt{d} (down) wird ein interner Counter hoch- bzw. runtergezählt. Arbeiten sie sich in den gegebenen Code ein und erstellen sie das Game-Loop
\begin{minted}{haskell}
mainLoop :: RWST Env () State IO ()
\end{minted}
und die Tasteneingabe
\begin{minted}{haskell}
getInput :: RWST Env () State IO Input
\end{minted}
Benutzen sie hierzu die gegebene pure Hilfsfunktion
\begin{minted}{haskell}
getInputfromEnv :: Char -> Env -> Input
\end{minted}

\end{document}