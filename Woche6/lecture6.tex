\documentclass{beamer}

\usepackage[utf8]{inputenc}
\usepackage[T1]{fontenc}
\usepackage[ngerman]{babel}
\usepackage{graphicx} % Bilder
\usepackage{wrapfig} % Umflussbilder
\usepackage{multicol} % Multiple columns
\usepackage{minted} % Haskell source code
\usepackage{framed} % Frames around source code
\usepackage[framemethod=tikz]{mdframed} % Frames
\usepackage{verbatim} % \begin{comment}...\end{comment}
\usepackage{etoolbox} % manipulate minted
\AtBeginEnvironment{minted}{\fontsize{10}{10}\selectfont}
\AfterEndEnvironment{minted}{}

\mdfdefinestyle{fancy}{
  roundcorner=5pt,
  linewidth=4pt,
  linecolor=red!80,
  backgroundcolor=red!20
}
\newmdenv[style=fancy]{important}

% redifine \em for \emph to use bold instead of italics
\makeatletter
\DeclareRobustCommand{\em}{%
  \@nomath\em \if b\expandafter\@car\f@series\@nil
  \normalfont \else \bfseries \fi}
\makeatother

% Stuff for Beamer
\beamertemplatenavigationsymbolsempty
\usetheme{Warsaw}

\title{Fortgeschrittene Funktionale Programmierung in Haskell}

\begin{document}
  
%----------------------------------------------------------------------------------------  

  \begin{frame}
  \begin{center}
    \huge\textbf{Fortgeschrittene Funktionale Programmierung in Haskell}\\ \bigskip
    \LARGE Universität Bielefeld, Sommersemester 2015\\ \bigskip
    \large Jonas Betzendahl \& Stefan Dresselhaus
    \end{center}
  \end{frame}

%----------------------------------------------------------------------------------------  
\begin{frame}[allowframebreaks]{Outline}
\frametitle{Übersicht}
\tableofcontents
\end{frame}

\section{Übersicht}

%----------------------------------------------------------------------------------------

\begin{frame}[fragile]

\Large
\textbf{Leseempfehlung:}
\normalsize

\begin{multicols}{2}
\includegraphics[scale=0.45]{parcur.png} 
\columnbreak

Wunderbares Buch zum Thema von Simon Marlow.\pause\bigskip

Nicht in der Uni-Bibliothek, dafür aber Gratis im Internet verfügbar, inklusive Beispielcode auf \texttt{Hackage}.
\end{multicols}

\end{frame}

\subsection{Motivation}

%----------------------------------------------------------------------------------------

\begin{frame}[fragile]

\begin{center}
\Large
\textbf{Motivation}
\end{center}

\end{frame}

%----------------------------------------------------------------------------------------

\begin{frame}[fragile]

\begin{center}
\huge
\emph{Free Lunch is over!}\bigskip

\normalsize
Herb Sutter (2005)
\end{center}
\pause
Die Hardware unserer Computer wird seit mehreren Jahren schon schneller breiter (\emph{mehr} Kerne) als tiefer (\emph{schnellere} Kerne).\pause\smallskip

Um technischen Fortschritt voll auszunutzen ist es also essentiell, gute Werkzeuge für einfache und effiziente Parallelisierung bereit zu stellen.
\end{frame}

%----------------------------------------------------------------------------------------

\subsection{Definitionen}

\begin{frame}[fragile]

\begin{center}
\Large
\textbf{Definitionen}
\end{center}

\end{frame}

%----------------------------------------------------------------------------------------

\begin{frame}
\underline{Parallelism vs. Concurrency:}\smallskip

Beides ist ein Ausdruck von \glqq Dinge gleichzeitig tun\grqq ; in der Programmierung haben sie aber grundverschiedene Bedeutungen.\pause\bigskip

Programme arbeiten \emph{parallel}, wenn sie mehrere Prozessorkerne einsetzen, um schneller an die Antwort einer bestimmten Frage zu kommen.\pause\bigskip

\emph{Nebenläufige} Programme hingegen haben mehrere \glqq threads of control\grqq . Oft dient das dazu, gleichzeitig mit mehreren externen Agenten (dem User, einer Datenbank, \dots) zu interagieren.
\end{frame}

%----------------------------------------------------------------------------------------

\begin{frame}
More foo about parallelism and determinism and such...
\end{frame}

%----------------------------------------------------------------------------------------

\begin{frame}
\underline{(WH)NF:}\smallskip

Im Themenbereich Parallelism wird oft darüber gesprochen, wann Ausdrücke ausgewertet werden und \glqq wie weit\grqq\ (Laziness). Es gibt dafür zwei wichtige Vokabeln:
\textbf{Normal Form} und \textbf{Weak Head Normal Form}.\bigskip\pause

Die \textbf{NF} eines Ausdrucks ist der gesamte Ausdruck, vollständig berechnet. Es gibt keine Unterausdrücke, die weiter ausgewertet werden könnten.\bigskip\pause

Die \textbf{WHNF} eines Ausdrucks ist der Ausdruck, evaluiert zum äußersten Konstruktor oder zur äußersten $\lambda$-Abstraktion (dem \emph{head}). Unterausdrücke können berechnet sein oder auch nicht. Ergo ist jeder Ausdruck in \textbf{NF} auch in \textbf{WHNF}.
\end{frame}

\begin{frame}[fragile]

\underline{\textbf{(WH)NF} Zuschauer-Wachzustands-Überprüfungs-Quiz:}\smallskip

Sind diese Ausdrücke in \textbf{NF} oder \textbf{WHNF}? Wenn ja welche davon?
\bigskip\pause

\mint{haskell}|(1337, "Hello World!")|
\pause
$\Rightarrow$ \textbf{NF} und \textbf{WHNF}! Der komplette Ausdruck ist evaluiert.
\pause

\mint{haskell}|\x -> 2 + 2 |
\pause
$\Rightarrow$ \textbf{WHNF}! Der \emph{head} ist eine $\lambda$-Abstraktion.
\pause

\mint{haskell}|'f' : ("oo" ++ "bar")|
\pause
$\Rightarrow$ \textbf{WHNF}! Der \emph{head} ist der Konstruktor \texttt{(:)}.
\pause

\mint{haskell}|(\x -> x + 1) 2|
\pause
$\Rightarrow$ Weder noch! Äußerster Part ist Anwendung der Funktion.

\end{frame}

%----------------------------------------------------------------------------------------

\subsection{Technisches}

\begin{frame}[fragile]
\underline{Ein paar technische Feinheiten:}\pause\bigskip

Um Programme in \texttt{Haskell} parallel ausführen zu können, müssen sie wie folgt
kompiliert werden:\smallskip

\texttt{\$ ghc --make -rtsopts -threaded Main.hs}
\pause
\bigskip

Danach können sie auch mit RTS (Run Time System) - Optionen wie z.B. diesen hier ausgeführt werden:\smallskip

\texttt{\$ ./Main.hs +RTS -N2 -s -RTS}
\bigskip
\pause

Dokumentation findet sich leicht via beliebiger Suchmaschine.

Eine Kurzübersicht gibt es zum Beispiel unter \texttt{cheatography.com/nash/cheat-sheets/ghc-and-rts-options/}

\end{frame}


%----------------------------------------------------------------------------------------
\section{Parallelism}

\begin{frame}

\begin{center}
\Large
\textbf{Parallelism}\normalsize\bigskip

\begin{itemize}
\item Die \texttt{Eval}-Monade und Strategies
\item Die \texttt{Par}-Monade
\item Die \texttt{RePa}-Bibliothek
\item GPU-Programming mit \texttt{Accelerate}
\end{itemize}
\end{center}

\end{frame}

%----------------------------------------------------------------------------------------

\subsection{Die Eval-Monade und Strategies}

\begin{frame}[fragile]

\begin{center}
\Large
\textbf{Parallelism}\normalsize\bigskip
\begin{itemize}
\item $\circ$ Die \texttt{Eval}-Monade und Strategies
\item Die \texttt{Par}-Monade
\item Die \texttt{RePa}-Bibliothek
\item GPU-Programming mit \texttt{Accelerate}
\end{itemize}
\end{center}

\end{frame}

%----------------------------------------------------------------------------------------

\begin{frame}[fragile]

Das Modul \texttt{Control.Parallel.Strategies} (aus dem Paket \texttt{parallel}) stellt uns
die \texttt{Eval}-Monade und einige Funktionen vom Typ \emph{Strategy} zur Verfügung, \dots\pause

\mint{haskell}|    type Strategy a = a -> Eval a|
\pause

\dots insbesondere die Strategies \texttt{rpar} und \texttt{rseq}. Dazu gleich mehr.\pause\bigskip

Desweiteren stellt es die Operation \texttt{runEval}, die die monadischen 
Berechnungen ausführt und das Ergebnis zurück gibt, bereit.

\mint{haskell}|    runEval :: Eval a -> a|
\pause
\bigskip

Wohlgemerkt: \texttt{runEval} ist \emph{pur!}

Wir müssen nicht gleichzeitig auch in der \texttt{IO}-Monade sein.

\end{frame}

%----------------------------------------------------------------------------------------

\begin{frame}[fragile]
\texttt{rpar} ist die Strategie, die ihr Argument parallel auswertet und währenddessen das Programm weiter laufen lässt.
\bigskip\pause

\texttt{rseq} ist die Strategie, die auf das Ergebnis ihres Argumentes wartet und erst dann mit dem Programm weiter macht.
\bigskip\pause

\emph{Protips:}\pause
\begin{itemize}
\item Ausgewertet wird jeweils zur WHNF (wenn nichts anderes angegeben wurde).\pause
\item Wird \texttt{rpar} ein bereits evaluierter Ausdruck übergeben, passiert nichts,
weil es keine Arbeit parallel auszuführen gibt. \pause
\end{itemize}
\bigskip

Sehen wir uns das mal \emph{in action} an\dots
\end{frame}


%----------------------------------------------------------------------------------------

\begin{frame}[fragile]
\underline{Ein Beispiel (1):}\smallskip

Wir wollen die Ausdrücke \texttt{(f x)} und \texttt{(f y)} mit der \texttt{Eval}-Monade parallel auswerten. O.B.d.A. benötigt \texttt{(f x)} mehr Zeit.\pause

\begin{multicols}{2}
\begin{minted}{haskell}
-- don't wait for evaluation
runEval $ do
    a <- rpar (f x)
    b <- rpar (f y)
    return (a,b)
\end{minted}
%$
\columnbreak
\pause
\includegraphics[scale=0.7]{evalmonad_01.png}
\end{multicols}
\pause

Hier passiert das \texttt{return} sofort. Der Rest des Programmes läuft weiter, während \texttt{(f x)} und \texttt{(f y)} (parallel) ausgewertet werden.
\end{frame}

%----------------------------------------------------------------------------------------

\begin{frame}[fragile]
\underline{Ein Beispiel (2):}\smallskip

Wir wollen die Ausdrücke \texttt{(f x)} und \texttt{(f y)} mit der \texttt{Eval}-Monade parallel auswerten. O.B.d.A. benötigt \texttt{(f x)} mehr Zeit.

\begin{multicols}{2}
\begin{minted}{haskell}
-- wait for (f y)
runEval $ do
    a <- rpar (f x)
    b <- rseq (f y) -- wait
    return (a,b)
\end{minted}
%$
\columnbreak
\pause
\includegraphics[scale=0.7]{evalmonad_02.png}
\end{multicols}
\pause

Hier werden \texttt{(f x)} und \texttt{(f y)} ebenfalls ausgewertet,
allerdings wird mit \texttt{return} gewartet, bis \texttt{(f y)} zu Ende evaluiert wurde.
\end{frame}


%----------------------------------------------------------------------------------------

\begin{frame}[fragile]
\underline{Ein Beispiel (3):}\smallskip

Wir wollen die Ausdrücke \texttt{(f x)} und \texttt{(f y)} mit der \texttt{Eval}-Monade parallel auswerten. O.B.d.A. benötigt \texttt{(f x)} mehr Zeit.

\begin{multicols}{2}
\begin{minted}{haskell}
-- wait for (f y) and (f x)
runEval $ do
    a <- rpar (f x)
    b <- rseq (f y) -- wait
    rseq a          -- wait
    return (a,b)
\end{minted}
%$
\columnbreak
\pause
\includegraphics[scale=0.7]{evalmonad_03.png}
\end{multicols}
\pause

In diesem Code wird sowohl auf \texttt{(f x)} als auch auf \texttt{(f y)} gewartet, 
bevor etwas zurück gegeben wird.
\end{frame}


%----------------------------------------------------------------------------------------

\begin{frame}[fragile]
\underline{Ein Beispiel (3):}\smallskip

Wir wollen die Ausdrücke \texttt{(f x)} und \texttt{(f y)} mit der \texttt{Eval}-Monade parallel auswerten. O.B.d.A. benötigt \texttt{(f x)} mehr Zeit.

\begin{multicols}{2}
\begin{minted}{haskell}
-- perhaps more readable:
runEval $ do
    a <- rpar (f x)
    b <- rpar (f y)
    rseq a          -- wait
    rseq b          -- wait
    return (a,b)
\end{minted}
%$
\columnbreak
\includegraphics[scale=0.7]{evalmonad_03.png}
\end{multicols}

In diesem Code wird sowohl auf \texttt{(f x)} als auch auf \texttt{(f y)} gewartet, 
bevor etwas zurück gegeben wird.
\end{frame}

%----------------------------------------------------------------------------------------

\subsection{Die Par-Monade}

\begin{frame}[fragile]

\begin{center}
\Large
\textbf{Parallelism}\normalsize\bigskip
\begin{itemize}
\item Die \texttt{Eval}-Monade und Strategies
\item $\circ$ Die \texttt{Par}-Monade
\item Die \texttt{RePa}-Bibliothek
\item GPU-Programming mit \texttt{Accelerate}
\end{itemize}
\end{center}

\end{frame}

%----------------------------------------------------------------------------------------

\subsection{Die RePa-Bibliothek}

\begin{frame}[fragile]

\begin{center}
\Large
\textbf{Parallelism}\normalsize\bigskip
\begin{itemize}
\item Die \texttt{Eval}-Monade und Strategies
\item Die \texttt{Par}-Monade
\item $\circ$ Die \texttt{RePa}-Bibliothek
\item GPU-Programming mit \texttt{Accelerate}
\end{itemize}
\end{center}

\end{frame}

%----------------------------------------------------------------------------------------

\subsection{Accelerate}

\begin{frame}[fragile]

\begin{center}
\Large
\textbf{Parallelism}\normalsize\bigskip
\begin{itemize}
\item Die \texttt{Eval}-Monade und Strategies
\item Die \texttt{Par}-Monade
\item Die \texttt{RePa}-Bibliothek
\item $\circ$ GPU-Programming mit \texttt{Accelerate}
\end{itemize}
\end{center}

\end{frame}

%----------------------------------------------------------------------------------------

\end{document}