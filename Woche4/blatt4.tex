\documentclass[a4paper,10pt]{scrartcl}
\usepackage[utf8]{inputenc}
\usepackage[ngerman]{babel}
\usepackage{amsmath}
\usepackage{amsfonts}
\usepackage{amsthm}
\usepackage{geometry}
\usepackage{multicol}
\usepackage{minted}
\geometry{a4paper,left=30mm,right=30mm, top=3cm, bottom=2cm} 

\newcommand{\underfat}[1]{\underline{\textbf{#1}}}
\newcommand{\theuebungszettel}{4}

\parindent0pt

\begin{document}

\begin{center}
  \begin{huge}
    \underfat{Fortgeschrittene funktionale}\\
    \underfat{Programmierung in Haskell}\\
  \end{huge}
\begin{LARGE}
\textbf{Übungszettel \theuebungszettel}
\end{LARGE}
\end{center}
\section*{Aufgabe \theuebungszettel.1:}
In der Vorlesung haben wir an einem Beispiel gelernt, wie Parsing funktioniert. Nun haben wir aber amerikanische Freunde, die ein anderes Format benutzen:
\begin{verbatim}
154.41.32.99 29/06/2013 15:32:23 4 internet
76.125.44.33 29/06/2013 16:56:45 3 noanswer
123.45.67.89 29/06/2013 18:44:29 4 friend
100.23.32.41 29/06/2013 19:01:09 1 internet
151.123.45.67 29/06/2013 20:30:13 2 internet
\end{verbatim}
Hierbei sind die Geräte als Zahlen codiert:
\begin{enumerate}
 \item Mouse
 \item Keyboard
 \item Monitor
 \item Speakers
\end{enumerate}
Passen sie den gegebenen Code so an, dass sie eine 2. Funktion \texttt{parseAmericanLog} schreiben, die das amerikanische Logformat in dieselben Datenstrukturen überführt. Die weiteren Angaben (internet, noanswer, friend) sollen zunächst ignoriert werden.
\section*{Aufgabe \theuebungszettel.2:}
Das Management hat entschieden, dass nun mehr Informationen zur Verfügung stehen sollen. Erweitern sie ihr Programm erneut um ein Feld \texttt{Maybe Source}, sodass die Definition der Datenstruktur nun wie folgt aussieht:
\begin{minted}{haskell}
data Source = Internet | Friend | NoAnswer deriving Show

data LogZeile = LogZeile Datum IP Geraet (Maybe Source) deriving Show
\end{minted}
Implementieren sie die Änderung in beiden Parsern.\\
\textbf{Hinweis:} Die Funktion \texttt{option} aus \texttt{Data.Attoparsec.Combinator} könnte hilfreich sein.
\section*{Aufgabe \theuebungszettel.3:}
Stellen sie alles, was sie bisher haben zu einer Binärdatei zusammen, die als erstes Argument eine Datei im \glqq normalen\grqq \ Format nimmt und als zweites Argument eine Datei im \glqq amerikanischen\grqq \ Format. Geben sie dann die gesamte Datenstruktur aus, in der beide Logdateien nach Datum/Uhrzeit sortiert vorkommen.\\
\textbf{Beispiel:}
\footnotesize
\begin{verbatim}
./parser "log.txt" "logAm.txt"

Right [LogZeile (Datum {tag = 2013-06-29, zeit = 11:16:23}) (IP 124 67 34 60) Keyboard Nothing,
LogZeile (Datum {tag = 2013-06-29, zeit = 11:32:12}) (IP 212 141 23 67) Mouse Nothing,
LogZeile (Datum {tag = 2013-06-29, zeit = 11:33:08}) (IP 212 141 23 67) Monitor Nothing,
LogZeile (Datum {tag = 2013-06-29, zeit = 12:12:34}) (IP 125 80 32 31) Speakers Nothing,
LogZeile (Datum {tag = 2013-06-29, zeit = 12:51:50}) (IP 101 40 50 62) Keyboard Nothing,
LogZeile (Datum {tag = 2013-06-29, zeit = 13:10:45}) (IP 103 29 60 13) Mouse (Just NoAnswer)]
\end{verbatim}
\normalsize
(Umbrüche Textbedingt)
\end{document}