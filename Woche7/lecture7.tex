\documentclass{beamer}

\usepackage[utf8]{inputenc}
\usepackage[T1]{fontenc}
\usepackage[ngerman]{babel}
\usepackage{graphicx} % Bilder
\usepackage{wrapfig} % Umflussbilder
\usepackage{multicol} % Multiple columns
\usepackage{minted} % Haskell source code
\usepackage{framed} % Frames around source code
\usepackage[framemethod=tikz]{mdframed} % Frames
\usepackage{verbatim} % \begin{comment}...\end{comment}
\usepackage{etoolbox} % manipulate minted
\AtBeginEnvironment{minted}{\fontsize{10}{10}\selectfont}
\AfterEndEnvironment{minted}{}

\mdfdefinestyle{fancy}{
  roundcorner=5pt,
  linewidth=4pt,
  linecolor=red!80,
  backgroundcolor=red!20
}
\newmdenv[style=fancy]{important}

% redifine \em for \emph to use bold instead of italics
\makeatletter
\DeclareRobustCommand{\em}{%
  \@nomath\em \if b\expandafter\@car\f@series\@nil
  \normalfont \else \bfseries \fi}
\makeatother

% Stuff for Beamer
\beamertemplatenavigationsymbolsempty
\usetheme{Warsaw}

\title{Fortgeschrittene Funktionale Programmierung in Haskell}

\begin{document}
  
%----------------------------------------------------------------------------------------  

  \begin{frame}
  \begin{center}
    \huge\textbf{Fortgeschrittene Funktionale Programmierung in Haskell}\\ \bigskip
    \LARGE Universität Bielefeld, Sommersemester 2015\\ \bigskip
    \large Jonas Betzendahl \& Stefan Dresselhaus
    \end{center}
  \end{frame}

%----------------------------------------------------------------------------------------  
\begin{frame}[allowframebreaks]{Outline}
\vfill
Übersicht für Heute:\smallskip

\tableofcontents
\end{frame}

\section{Wiederholung}

%----------------------------------------------------------------------------------------

\begin{frame}

\begin{center}
\Large
\textbf{Wiederholung}
\end{center}

\end{frame}

%----------------------------------------------------------------------------------------

\begin{frame}

\textbf{Leseempfehlung:}

\begin{center}
\includegraphics[scale=0.5]{../Woche6/parcur.png} 
\end{center}
\pause

\dots srsly!

\end{frame}

%----------------------------------------------------------------------------------------

\begin{frame}

\begin{center}
\includegraphics[scale=1]{../Woche6/parcur.png} 
\end{center}

\end{frame}

%----------------------------------------------------------------------------------------

\begin{frame}

\textbf{Überblick:}
\pause

\begin{multicols}{2}
\textbf{Parallelism:}
\begin{itemize}
\item Mehrere Hardwareelemente\pause
\item Antwort schneller kriegen\pause
\item deterministisch (i.d.R.)\pause
\item oft deklarativ\pause
\end{itemize}
\columnbreak
\textbf{Concurrency:}
\begin{itemize}
\item Mehrere Threads\pause
\item Dinge gleichzeitig tun\pause
\item nichtdeterministisch\pause
\item oft impertativ
\end{itemize}
\end{multicols}

\end{frame}

%----------------------------------------------------------------------------------------

%----------------------------------------------------------------------------------------

\section{Threads, MVars, etc.}

\begin{frame}

\begin{center}
\Large
\textbf{Die Basics: Threads, MVars, etc.}
\end{center}

\end{frame}

%----------------------------------------------------------------------------------------

\begin{frame}[fragile]

Wir beginnen mit der Funktion, die einen neuen Thread erstellt:

\mint{haskell}|forkIO :: IO () -> IO ThreadId|
\pause

Threads interagieren notwendigerweise mit der Welt, ergo ist die Berechnung, die wir 
übergeben vom Typ \texttt{IO ()}.\smallskip\smallskip
\pause

Die \texttt{ThreadId} kann später benutzt werden um z.B. den Thread vorzeitig zu töten oder ihm
eine Exception zuzuschmeißen.
\end{frame}

%----------------------------------------------------------------------------------------

\begin{frame}[fragile]

Ein kleines Beispiel:

\begin{minted}{haskell}
import Control.Concurrent
import Control.Monad
import System.IO

main :: IO ()
main = do
  hSetBuffering stdout NoBuffering
  forkIO (replicateM_ 100000 (putChar 'A')) 
  replicateM_ 100000 (putChar 'B')
\end{minted}
\pause
\dots Output?
\pause
\begin{verbatim}
AAAAAAAAABABABABABABABABABABABABABABABABABABABABABABAB
ABABABABABABABABABABABABABABABABABABABABABABABABABABAB
ABABABABABABABABABABABABABABABABABABABABABABABABABABAB
ABABABABABABABABABABABABABABABABABABABABABABABABABABAB
\end{verbatim}
\end{frame}

%----------------------------------------------------------------------------------------

\begin{frame}[fragile]

Aber\dots \pause wie kriegen wir jetzt Ergebnisse aus der Berechnung raus?\\
Der Typ ist nur \texttt{IO ()}, das liefert nichts (interessantes) zurück!\pause\bigskip

Das gleiche Problem hatten wir schon in der \texttt{Par}-Monade. Lösung damals waren
\texttt{IVar}s:\bigskip

\begin{minted}{haskell}
data IVar a  -- instance Eq

new :: Par (IVar a)
put :: NFData a => IVar a -> a -> Par ()
get :: IVar a -> Par a
\end{minted}
\end{frame}

%----------------------------------------------------------------------------------------

\begin{frame}[fragile]

Introducing: \dots \pause \texttt{MVar}s!\bigskip

\begin{minted}{haskell}
data MVar a  -- abstract

newEmptyMVar :: IO (MVar a)
newMVar      :: a -> IO (MVar a)
takeMVar     :: MVar a -> IO a
putMVar      :: MVar a -> a -> IO ()

readMVar     :: MVar a -> IO a
\end{minted}
\pause

Wir brauchen hier keine eigene Monade wie \texttt{Par}. Da Concurrency so oder so
effektvoll ist, reicht \texttt{IO} vollkommen aus.\bigskip

Unterschied zwischen \texttt{IVar}s und \texttt{MVar}s: erstere sind \emph{\textbf{i}}mmutable,
letztere sind \emph{\textbf{m}}utable.

\end{frame}

%----------------------------------------------------------------------------------------

\begin{frame}[fragile]

Ein Beispiel zu \texttt{MVar}s:

\begin{minted}{haskell}
main :: IO ()
main = do
  m <- newEmptyMVar
  forkIO $ do putMVar m 'x'; putMVar m 'y'
  r <- takeMVar m
  print r
  r <- takeMVar m
  print r
\end{minted}
%$
\pause

Wie wir sehen kann die gleiche \texttt{MVar} über Zeit mehrere Zustände annehmen
und erfolgreich zur Kommunikation zwischen Threads benutzt werden. 

\end{frame}

%----------------------------------------------------------------------------------------

\begin{frame}
Generell haben \texttt{MVar}s drei Hauptaufgaben:\pause

\begin{itemize}
\item \textbf{Channel mit nur einem Slot}\\
Eine \texttt{MVar} kann als Nachrichtenkanal zwischen Threads benutzt werden, allerdings maximal eine Nachricht auf einmal halten.\pause

\item \textbf{Behältnis für shared mutable state}\\
In Concurrent Haskell brauchen oft mehrere Threads Zugriff auf einen shared state. Ein beliebtes Designpattern ist, das dieser State als normaler (immutable) Haskell-Datentyp repräsentiert und in einer \texttt{MVar} verpackt wird.\pause

\item \textbf{Baustein für kompliziertere Strukturen}
\end{itemize}
\end{frame}

%----------------------------------------------------------------------------------------

\begin{frame}[fragile]

\textbf{Mehr Leckerlis:}\smallskip\smallskip

Was passiert, wenn wir folgenden Code ausführen?

\begin{minted}{haskell}
main :: IO ()
main = do m <- newEmptyMVar
          takeMVar m
\end{minted}
\pause
\bigskip

Wir bekommen eine Fehlermeldung, dass das Programm hängt, statt einfach nur ein hängendes Programm.

\begin{verbatim}
$ ./mvar3
mvar3: thread blocked indefinitely in an MVar operation
\end{verbatim}
\end{frame}

%----------------------------------------------------------------------------------------

\begin{frame}
\textbf{Deadlock detection:}

\begin{multicols}{2}
Threads und \texttt{MVar}s sind Objekte auf dem Heap. Das \texttt{RTS} (i.e. der Garbage collector) durchläuft den Heap um alle lebendigen Objekte zu finden, angefangen bei den laufenden Threads und ihren Stacks.\smallskip

Alles was so nicht erreichbar sind (z.B. ein Thread der auf eine \texttt{MVar} wartet, die nirgendwo sonst referenziert wird), blockiert und bekommt eine Exception geschmissen.

\columnbreak

\begin{figure}
\includegraphics[scale=0.52]{dining_philosophers.png} 
\caption{dining philosophers}
\end{figure}

\end{multicols}
\end{frame}

%----------------------------------------------------------------------------------------

\begin{frame}[fragile]

\textbf{Deadlock detection:}\smallskip\smallskip

Dieses Vorgang funktioniert allerdings nicht immer wie man zunächst denkt.
Beispiel: Was passiert mit diesem Code?

\begin{minted}{haskell}
main :: IO ()
main = do 
  lock <- newEmptyMVar
  complete <- newEmptyMVar
  forkIO $ takeMVar lock `finally` putMVar complete ()
  takeMVar complete
\end{minted}
%$
\pause
\bigskip

Da nicht nur der geforkte Thread sondern auch der ursprüngliche gedeadlocked sind, wird hier die Fehlermeldung geprintet, statt die rettende Exception an das Kind zu sende.

\end{frame}

%----------------------------------------------------------------------------------------

\section{Software Transactional Memory}

\begin{frame}

\begin{center}
\Large
\textbf{Software Transactional Memory (STM)}
\end{center}

\end{frame}

%----------------------------------------------------------------------------------------

\section{Parallelism through concurrency}

\begin{frame}

\begin{center}
\Large
\textbf{Parallelism through Concurrency}
\end{center}

\end{frame}

%----------------------------------------------------------------------------------------

\section{Distributed Programming}

\begin{frame}

\begin{center}
\Large
\textbf{Distributed Programming}
\end{center}

\end{frame}

\end{document}
